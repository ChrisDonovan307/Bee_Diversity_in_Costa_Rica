\begin{table}[!h]
\centering
\caption{\label{tab:phyto_table}Phytogeographic units represented in sampled farms}
\centering
\begin{tabular}[t]{cp{5in}}
\hline\hline
Unit & Name\\
\midrule
3.2 & Bosque siempreverde de tierras elevadas de las llanuras de Tortuguero\\
6.2 & Bosque deciduo a semideciduo de las llanuras de Guanacaste-Valle Central e Islas del Pacífica Norte\\
9.1 & Bosque semideciduo a siempreverde estacional de la Vertiente Pacífica de la Cordillera de Tilarán\\
10.1 & Bosque siempreverde estacional de la Vertiente Pacífica de la Cordillera Central\\
10.2 & Bosque siempreverde de la Vertiente Caribe de la Cordillera Central\\
11.1 & Bosque mixto intermedio estacional, subnuboso a nuboso de la Vertiente Pacífica de la Cordillera de Talamanca\\
11.3 & Bosque mixto intermedio siempreverde, subnuboso a nuboso de la Vertiente Caribe de la Cordillera de Talamanca\\
12.1 & Bosque siempreverde estacional de las serranías y valles del Pacífico Central\\
12.2 & Bosque siempreverde subnuboso a nuboso de las serranías del Pacífico Central\\
14.1 & Bosque siempreverde estacional subnuboso a nuboso de la Cordillera o Fila Costeña Norte\\
15.1 & Bosque semideciuo estacional del Valle de El Genera\\
\hline\hline
\end{tabular}
\end{table}
